%
% $Header: /cygdrive/C/CVSNT/cvsroot/Cattaneo/Relazioni/CV/CV2.tex,v 1.1 2010-05-03 07:29:38 cattaneo Exp $
%

% $Log: not supported by cvs2svn $
% Revision 1.21  1999/04/15 16:08:49  cattaneo
% *** empty log message ***
%
% Revision 1.20  1999/04/14 09:57:26  cattaneo
% *** empty log message ***
%
% Revision 1.19  1999/04/14 09:46:15  cattaneo
% *** empty log message ***
%
% Revision 1.18  1999/04/08 18:01:41  cattaneo
% *** empty log message ***
%
% Revision 1.18  1999/04/08 18:00:16  cattaneo
% Nuova versione per concorso 1999
%
% Revision 1.3  1999/04/06 06:05:48  cattaneo
% *** empty log message ***
%
% Revision 1.2  1999/04/06 03:40:30  cattaneo
% *** empty log message ***
%
% Revision 1.17  1997/07/14 08:27:51  cattaneo
% *** empty log message ***
%
% Revision 1.16  1997/07/07 16:38:52  cattaneo
% *** empty log message ***
%
% Revision 1.15  1996/07/10  16:05:35  cattaneo
% *** empty log message ***
%
% Revision 1.14  1996/07/10  15:11:47  cattaneo
% *** empty log message ***
%
% Revision 1.13  1996/07/09  15:00:48  cattaneo
% *** empty log message ***
%
% Revision 1.12  1996/07/09  11:16:56  cattaneo
% *** empty log message ***
%
% Revision 1.11  1996/07/09  11:09:38  cattaneo
% *** empty log message ***
%
% Revision 1.10  1996/07/09  09:30:19  cattaneo
% 


\documentclass[11pt]{article}
\usepackage[italian]{babel}
\usepackage{epsfig}
\usepackage{fancyheadings}
\usepackage{fancybox}


% riaggiustamenti alla traduzione di babel
\addto\captionsitalian{%
  \def\contentsname{Sommario}
  \def\listfigurename{Lista delle figure}
  \def\listtablename{Lista delle tavole}
  \def\refname{Elenco delle pubblicazioni}
  \def\indexname{Indice}
  \def\figurename{Figura}
  \def\tablename{Tavola}
  \def\appendixname{Appendice}
}
%% DIMENSIONI della PAGINA:


% Paragraphs are marked by large space rather than indentation:
\parindent 0pt
\parskip 7pt plus 1pt minus 1pt
\setlength{\textwidth}{15.7cm}
\setlength{\oddsidemargin}{0cm}      % this implies 1 inch margin
\setlength{\evensidemargin}{0cm}     % TeX Built-in
\setlength{\textheight}{22.5cm}      % height of text on page

\makeatletter

\newlength{\posii}
\newlength{\posiibox}
\newlength{\posiil}
\newlength{\finepar}
\newlength{\fineitem}
\newlength{\boxsize}

\setlength{\posii}{3.8cm}
\setlength{\posiil}{4.5cm}
\setlength{\finepar}{55pt}
\setlength{\fineitem}{10pt}
\setlength{\boxsize}{\textwidth}
\addtolength{\boxsize}{-1\posii}
\setlength{\posiibox}{\posii}
\addtolength{\posiibox}{-15pt}

% parametri per i box (package fancybox)
\setlength{\fboxsep}{10pt}
\newlength{\titlewidth}
\setlength{\titlewidth}{\textwidth}
\addtolength{\titlewidth}{-2\fboxsep}
\newlength{\figwidth}
\setlength{\figwidth}{\textwidth}
\addtolength{\figwidth}{-2\fboxsep}
\addtolength{\figwidth}{-10pt}

% parametri per gli header e footer (package fancyheader)
\setlength{\headrulewidth}{1 pt}
\setlength{\footrulewidth}{1 pt}
\setlength{\plainheadrulewidth}{0pt}
\setlength{\plainfootrulewidth}{0pt}
\setlength{\headsep}{10 mm}   % spazio dall'  header e l' inizio del testo
\setlength{\topmargin}{-20mm} % 0cm  lascia un margine di 2 cm su A4
                                     % a questo si somma \headheight
\setlength{\headheight}{20mm} % altezza dell'  header. 0 => no header
\setlength{\topskip}{0mm}
\setlength{\footskip}{15mm}   % distanza dal testo al footer



%% DATES, VERSIONS AND TITLES:

\def\fileversion{$Revision: 1.1 $}
\def\filedate{$Date: 2010-05-03 07:29:38 $}

% N.B. date nel formato italiano 
\def\today{\number\day \space\ifcase\month\or
  Gennaio\or Febbraio\or Marzo\or Aprile\or Maggio\or Giugno\or
  Luglio\or Agosto\or Settembre\or Ottobre\or Novembre\or Dicembre\fi
  \space \number\year}

\def\expanddate $#1:#2/#3/#4 #5:#6:#7${\day=#4 \month=#3 \year=#2}
\begingroup
  \expandafter\expanddate\filedate
  \xdef\thefiledate{\today}
\endgroup

\def\expandrevision $#1:#2.#3${Versione: #2.#3}
\begingroup
  \xdef\thefileversion{\expandafter\expandrevision\fileversion}
\endgroup


%
% First page parameters ..
%
\setcounter{page}{1}
\newcounter{lastpage}

\author{\Large \sl Giuseppe CATTANEO}

\date{\thefileversion \\ del \thefiledate}

\def\headtitle{Curriculum dell'attivit\`a scientifica e
didattica di {\sl Giuseppe CATTANEO}}

\newenvironment{bigtitle} {
  \fontencoding{OT1}\fontfamily{cmfib}
  \fontseries{m}
  \fontshape{n}
  \fontsize{20}{25pt}\selectfont}{}



\title{{\sffamily\bfseries {\Huge Curriculum\\}   
 {\Large dell'attivit\a`a scientifica e didattica\\}}}

\def\@maketitle{%
  \begin{Sbox}
    \begin{minipage}{\titlewidth}
      \begin{center}
           \@title
               \vskip 1.5em
               {\begin{tabular}[t]{c}\@author\end{tabular}\par}
        \vskip 1.5em{\small\@date}%
      \end{center}
   \end{minipage}
  \end{Sbox}
  \begin{center}
    \shadowbox{\TheSbox}
  \end{center}
  \par
}


%% TABLE OF CONTENTS

\newskip\myskip

\def\tableofcontents{%
  \par\vfill
  \begin{quote}
    \begin{center} \huge\bf \contentsname \end{center}
    \def\numberline##1{\hbox to 0pt{\hss##1\hskip 1em}}%
    \let\oldaddvspace\addvspace
    \def\addvspace##1{%
      \myskip##1\relax
      \oldaddvspace{.5\myskip}}
    \@starttoc{toc}%
  \end{quote}
  \thispagestyle{empty}
  \vfill
  \clearpage}

%% PAGE STYLE:

\pagestyle{fancy}

\if@twoside
  \markboth{\headtitle}%
    {\headtitle}%
\else
  \markright{\headtitle}
\fi
\lhead{}
\chead{\rm \small \sl \headtitle}
\rhead{}

\lfoot{\rm \small \sl \thefileversion\  del \thefiledate}
\cfoot{}
\rfoot{\rm \small \sl -- \thepage\ / \pageref{lastpage}\ --} % page number

%
% -------------- fine definizioni ------------------------
%

\makeatother

\begin{document}

% prima pagina
\maketitle
\newpage
\tableofcontents

%
% ---------------------------------------------------
%

\section{Dati Anagrafici}
\begin{tabbing}
 \makebox[\posii]{}\=\makebox[\boxsize]{}\=\kill
        {\sl Cognome:}\>        CATTANEO\\[\fineitem]
        {\sl Nome:}\>   Giuseppe\\[\fineitem]
        \parbox[t]{\posii}
        {\sl Data di nascita:}\> 11 Gennaio 1960\\[\fineitem]
        {\sl Luogo di nascita:}\> Bari\\[\fineitem]
        {\sl Residenza:}\> Via Panoramica, 15 -
                           I-84100 Salerno (SA)\\[\fineitem]
        {\sl Cittadinanza:}\> Italiana\\[\fineitem]
        {\sl Stato civile:}\> Coniugato con un figlio a carico\\[\fineitem]
        {\sl Telefono:}\> +39 089 792672\\[\fineitem]
        {\sl Ufficio:}\> +39 089 965330\\[\fineitem]
        {\sl FAX:}\>    +39 089 965272\\[\fineitem]
	{\sl E-Mail:} \> {\tt cattaneo@dia.unisa.it}\\
\end{tabbing}

% Posizione obblighi militari:  Esentato ai sensi dell'art. 23-ter D.L. 27/2/1982 n. 57
% Lingue straniere conosciute:  Inglese e Francese.


\section{Formazione}
\begin{description}
\item [[ 1978]]
\hfill \parbox[t]{\boxsize}     { Consegue la maturit\a`a scientifica
presso il Liceo Scientifico Statale ``F.~Severi'' con votazione
60/60.}

\item[[ 12/1983]] \hfill \parbox[t]{\boxsize}{Consegue la Laurea in
Scienze dell'Informazione con lode presso l'Universit\a`a degli studi
di Salerno, discutendo la tesi {\sl ``Architetture Special Purpose per
l'Elaborazione di Immagini\/''}.}
\end{description}

\section{Borse di studio}
\label{borse}
\begin{description}
\item [[ 1986]]
\hfill  \parbox[t]{\boxsize} {Ottiene una borsa di studio bandita
dall'Universit\a`a di Salerno con i fondi del FORMEZ, per un soggiorno
di 9 mesi presso un laboratorio di ricerca di un'Universit\a`a estera.}
\end{description}

\section{Servizi prestati nell'Ateneo Salernitano e in enti di ricerca}
\subsection{Posizione attuale}
 Ricercatore Universitario per il gruppo di discipline {\tt 92/bis}
 successivamente convertito in {\tt K05B} presso la Facolt\a`a di
 Scienze Matematiche Fisiche e Naturali dell'Universit\a`a degli studi
 di Salerno dal Maggio del 1986. Ricercatore confermato dal 26 Maggio 1989,
 afferisce al Dipartimento di Informatica ed Applicazioni ``{R.M. Capocelli}''.

\subsection{Precedenti esperienze}
\begin{description}
\item [[ 3/84 -- 5/86]]
\hfill  \parbox[t]{\boxsize} {Titolare di un contratto triennale
stipulato con l'Universit\a`a di Salerno ai sensi dell'art. 26 D.P.R.
380.}

\item [[ 10/86 -- 12/87]]
\hfill  \parbox[t]{\boxsize} {Soggiorno a Parigi
presso il Laboratoire d'Informatique Th\a'eorique et Programmation
(L.I.T.P.) Universit\a`a Parigi 6, nell'ambito di una collaborazione
scientifica tra il L.I.T.P. ed il Dipartimento di origine grazie ad una
borsa di studio di cui al punto \ref{borse}.}

\item [[ 7/88 -- 12/90]] \hfill \parbox[t]{\boxsize} {Contratto di
ricerca biennale ( n.871B00-7909245-LAAISLCIMAIA) con il L.I.T.P.
Universit\a`a Parigi 6 per lo sviluppo, la realizzazione e messa a
punto di una Lisp Machine (MAIA) interamente progettata e costruita in
Francia da un progetto del Centre National d'Etudes en
T\a'el\a'ecommunications (CNET).}
\end{description}

\section{Attivit\a`a Scientifica}

\subsection{Introduzione}
L'intera attivit\`a di ricerca \a`e stata schematizzata nelle
seguenti aree ed \a`e rappresentata graficamente rispetto agli anni di
attivit\a`a nella figura \ref{schema}:

\begin{enumerate}
\item Studio ed implementazione dei linguaggi di programmazione ed in
particolare dei linguaggi logico/funzionali.

\item Approccio al parallelismo mediante linguaggi funzionali.

\item Linguaggi Actor Oriented per il parallelismo massivo.

\item Progetto, Sperimentazione e Ingegnerizzazione di Algoritmi e
Strutture Dati.

\item Animazione di algoritmi e Computer Supported Cooperative
Workgroup on the {\sl web}.
\end{enumerate}

\begin{figure}[h]
  \framebox{
    \psfig{figure=Proget1.eps,width=\figwidth}
    }
  \caption{Svogimento delle attivit\a`a scientifiche in ordine cronologico}
\label{schema}
\end{figure}
  
  
La prima fase si \`e avviata negli anni successivi alla laurea.  In
questo periodo ha continuato gli studi intrapresi con la tesi nell'
ambito delle architetture parallele e lo sviluppo di linguaggi evoluti
\cite{ipl:1,icon:2}, cercando di accumulare esperienze nel campo delle
tecniche di programmazione per la realizzazione di grandi progetti
software.

Queste esperienze sono state poi utilizzate sia nello studio delle
tecniche  non convenzionali di implementazione di Prolog, sia
nell'utilizzo  del paradigma della programmazione
funzionale come approccio al parallelismo.

Grazie a ripetuti contatti con il prof. G. Agha, Direttore del ``{\sl
Open Systems Laboratory}'' presso l'Universit\`a dell'Illinois a
Urbana--Champaign (USA), l'approccio al parallelismo massivo si \`e
diretto verso i cos\a`\i\ detti  {\sl linguaggi ad attori}.  In
questo paradigma di programmazione confluiscono idee proprie della
programmazione funzionale, della programmazione Object Oriented e
perfino della programmazione logica ({\sl reflection}).

Nel 1993, in seguito alla diminuzione dell'interesse da parte
della  comunit\a`a scientifica nell'area del parallelismo massivo,
tutti gli  sforzi si sono concentrati sul progetto, analisi e
sperimentazione  di algoritmi su grafi dinamici, sfruttando una
proficua sinergia con il prof. G.F. Italiano.
Nel settore dell'{\sl algorithm engineering} sono stati
raggiunti risultati  di notevole interesse sia in campo applicativo
che  teorico.


\subsection{Approccio al parallelismo mediante linguaggi funzionali} 

Agli inizi del 1985 nell'ambito di una
collaborazione con l'unit\a`a del C.N.R. di Arco Felice (NA) 
ha partecipato in qualit\a`a di collaboratore esterno al
progetto strategico nazionale {\sl ``Tecnologia del Software''} con il gruppo
locale diretto dal dott. M. Furnari. In questo contesto 
ha avuto modo di avvicinarsi alle problematiche legate
all'implementazione ed alla definizione dei linguaggi logico
funzionali \cite{icon:2,mxlog:1} con particolari riferimenti al parallelismo
\cite{cnr:1,cnr:2}.

\label{parigi} 
Nell'ottobre del 1986 grazie ad una
borsa di studio per l'estero, si recava presso il Laboratoire
d'Informatique Th\a`eorique et Programmation dell'Universit\a`a Parigi 6
per interessarsi specificamente alle problematiche inerenti
l'implementazione dei linguaggi logico funzionali \cite{litp:1},
essendo tale laboratorio tra i pi\`u qualificati su questi temi.

Dopo un breve periodo di formazione, ha cominciato a partecipare alla
intensa attivit\a`a scientifica del laboratorio divenendo
``{\sl Chercheur Associ\a'e}'' del laboratorio e membro dell'equipe n.2
``{\sl Langages Applicatifs}'' di cui ha continuato  a far parte per
molti anni.
In questo periodo ha tenuto numerosi seminari per presentare gli stati
di avanzamento del progetto cui partecipava.

\a`E stato titolare di un contratto di
ricerca biennale (6/88--6/90) stipulato con il L.I.T.P
nell'ambito di una convenzione stipulata dal laboratorio con il
CNET per lo sviluppo e la valutazione di una LISP machine (MAIA),
interamente prodotta in Francia, da realizzarsi nel biennio 1988--90.
Il progetto globale si \a`e articolato su due sottoprogetti :

\begin{enumerate}
\item [a)] Sviluppo Sistema ISO Lisp sulla architettura MAIA.
\item [b)] Estensione del sistema Common Lisp con un insieme di
  primitive che potessero conciliare le esigenze di Multitasking e
  computazione simbolica con i vincoli di {\sl real-time} imposti
  dalle strumentazioni di controllo.
\end{enumerate}

Il contratto di ricerca di cui \`e stato titolare gli affidava il
coordinamento di una equipe di ricercatori per lo sviluppo del secondo
sottoprogetto, mentre la prima parte \a`e stata affidata ad un'altra equipe
diretta dal prof. C. Queinnec.  In tale ambito si \a`e arrivati alla
realizzazione di una delle prime implementazioni di un sistema
Common--LISP che forniva un insieme esteso di primitive per il
parallelismo a partire dal modello shared-memory e dalla proposta
di ``{\sl standard}'' presentata nel rapporto ``{\sl sceptre}''
\cite{litp:1,ISCIS:1}.

In questo stesso periodo, utilizzando ancora una volta una piattaforma
basata su sistemi Common--Lisp, ha attivamente collaborato con un'
equipe mista (italo-francese) allo studio ed alla realizzazione di un
sistema Prolog basato su una tecnica di implementazione assolutamente
innovativa rispetto a quelle correntemente impiegate per questo tipo
di implementazioni \cite{mxlog:2,SPLT:89,GULP:89}.  Questa tecnica
\a`e basata sulla possibilit\a`a di accedere e/o modificare lo stato
della computazione corrente sostanzialmente rappresentata come una
successione di contesti storici creati ad ogni chiamata funzionale.

Continuando lo studio sulle tecniche di implementazione di
linguaggi evoluti impiegando linguaggi di alto livello, senza
per questo rinunciare ad aspetti quali l'efficienza ed ottenendo,
d'altra parte, maggiore portabilit\a`a e manutenibilit\a`a, si \a`e
arrivati ad una versione finale del sistema Prolog che contiene,
oltre al dimostratore, anche un ambiente di programmazione
interamente basato su questa tecnica di implementazione 
\cite{ISCIS:2,SEKE:1}. In questo modo si \a`e ottenuto un
duplice risultato:

\begin{enumerate}
\item La dimostrazione della potenza e della flessibilit\a`a dello
  strumento utilizzato in un ambito le cui difficolt\`a d'implementazione
  sono ben note. Questo risultato scaturisce direttamente da estesi
  confronti con le tecniche standard ben consolidate.

\item In modo omogeneo si \a`e esteso il campo di applicazione di
  questa tecnica anche a settori della programmazione (Programming
  Environment) per i quali si riteneva non fosse possibile introdurre
  innovazioni sostanziali rispetto alle tecniche di implementazione
  gi\`a esistenti.
\end{enumerate}

Come ulteriore campo di indagine si \a`e indagato sulla possibilit\`a
di estendere il sistema Prolog realizzato con un compilatore che
utilizzi come codice intermedio un linguaggio evoluto (linguaggio C)
includendo tutti i costrutti che sono stati definiti per la
manipolazione della storia del calcolo.

A tal riguardo ha coordinato una tesi di dottorato presso il
laboratorio parigino che ha portato all'implementazione su scala
reale di un compilatore per un linguaggio Pascal like (CLIP) che rende
disponibili tutti i costrutti dedicati alla manipolazione della storia
del calcolo con la massima efficienza raggiungibile.  In seguito ad
ulteriori analisi questa tecnica si \a`e rivelata ugualmente utile
come strumento formale per la specifica funzionale di sistemi
complessi.


\subsection{Linguaggi Actor Oriented per il parallelismo massivo}

Dopo il rientro in Italia (1/90) ha continuato ad interessarsi a
problemi connessi al parallelismo ed alla programmazione evoluta.
Nell'ambito del progetto finalizato C.N.R. ``Calcolo Parallelo'' ha
analizzato nuove tecniche di implementazione per linguaggi fortemente
paralleli. Seguendo l'approccio formale di C. Hewitt sono state
investigate le aree dei cosidetti ``{\sl linguaggi ad attori}'' questa
volta in modo indipendente dalle architetture hardware e basandosi
sul paradigma di computazione ``{\sl Message Passing}''
\cite{ISCIS:3,ISCIS:4}.

\a`E stato realizzato un prototipo di un kernel minimale di un
linguaggio Actor-Based in collaborazione con il prof. M. Di Santo ed
il prof. G. Iannello del Dipartimento di Ingegneria Informatica e
Matematica dell'Universit\a`a di Salerno, che opera su una macchina
multiprocessore Encore Multimax e sfrutta il concetto di esecuzione
multithreads \cite{threads:1} per incapsulare l'attivit\a`a di ogni
attore, prescindendo dall'architettura sottostante (memoria
condivisa).  Dopo una fase di valutazione delle prestazioni e di messa
a punto del prototipo, il progetto \`e migrato su di una macchina
``{\sl Transputer-based}'' a pi\a`u alto grado di parallelismo
\cite{AICA:1,AICA:2}.


\subsection{Progetto, Sperimentazione e Ingegnerizzazione di Algoritmi e Strutture Dati.}

Nel corso del 1993, visto il crescente interesse che si andava
riversando verso l'area del {\em software engineering} e sfruttando
l'occasione di collaborare con il prof.  G.F. Italiano che nel 1994
prendeva servizio presso la stessa Facolt\a`a, il sottoscritto ha
cominciato ad avvicinarsi all'area algoritmica specializzandosi nel
campo della sperimentazione ed analisi di algoritmi.  Tale interesse
nasceva da diverse opportunit\a`a, seguendo le indicazioni provenienti
da pi\a`u settori della Computer Science, che tuttora tendono sempre
pi\a`u a legare i risultati scientifici (algoritmi e complessit\a`a
teoriche) con le esigenze concrete degli sviluppatori di applicazioni.

In tale lavoro sono confluite tutte le esperienze accumulate nello
sviluppo di progetti software di notevoli dimensioni, ed in
particolare sin dai primi risultati si \a`e cercato di mantenere uno
stretto contatto con gli altri gruppi di ricerca impegnati su temi
affini per poter confrontare i risultati della sperimentazione,
renderli omogenei e diffonderli tra i vari gruppi aumentando la
ricaduta nei settori di mercato adiacenti.
 
Il primo obiettivo di questa ricerca \`e stato quello di progettare e
realizzare  una comune piattaforma software per l'analisi sperimentale
di algoritmi che potesse avere la massima diffusione grazie
all'impiego  di strumenti standard quali C++ e la libreria di tipi di dati
astratti  LEDA, Library of Efficient Data types and Algorithms
sviluppata presso MPI a Saarbr\"ucken dai proff. K.  Melhorn e S. N\"aher.

Gli obiettivi e le fasi intermedie di questo progetto di ricerca sono
stati infatti definiti insieme agli stessi autori di LEDA ricadendo in
un comune progetto finanziato in parte con i fondi della Commissione
della  Comunit\`a Europea (progetto ESPRIT-LTR n.20244 ALCOM-IT) in
parte  dal MURST (progetto ``Efficienza di Algoritmi e Progetto di
Strutture  Informative''), e comprendono attivit\`a di tipo sia
teorico  che sperimentale.

Il sottoprogetto sviluppato a Salerno si \`e sviluppato in tre fasi:
\begin{itemize}
\item La prima \`e stata dedicata alla definizione ed all'acquisizione
degli strumenti di base e dell'ambiente operativo (Software di base,
Software di sviluppo, ed utility in generale) necessari per sviluppare
una piattaforma comune tra i vari gruppi di ricerca in modo da rendere
omogenei i dati sperimentali rilevati.

\item La seconda fase ha prodotto uno schema di valutazione delle
  prestazioni per algoritmi dinamici su grafi. In sostanza sono state
  messi a punto le procedure di benchmark e gli intervalli di
  variabilit\`a entro i quali valutare le prestazioni rispetto alle
  reali  esigenze che potenzialmente potevano emergere da casi reali di
  studio  ed applicazioni.

\item La terza fase ha prodotto l'implementazione di un insieme di algoritmi
  noti in letteratura e mai precedentemente analizzati dal
  punto di vista pratico. Una volta terminata la fase implementativa
  si \a`e passati all'analisi delle prestazioni dei vari casi di studio
  mediante  confronti diretti sia nel caso medio che nei casi estremi
  (worst case). 
\end{itemize}

Dal punto di vista algoritmico il lavoro si \a`e concentrato nell'area
dei  grafi dinamici ed in particolare su problematiche legate alla
connettivit\`a e al mantenimento del {\sl Minimum Spanning Tree}.

Nell'ambito della connettivit\`a su grafi dinamici sono state
confrontate le due tecniche pi\a`u famose, note con il nome di
``Sparsification'' di G.F. Italiano et al. e ``Dynamic Connectivity''
di M.R.  Henzinger.  Ricercando il massimo rapporto tra genericit\a`a
ed efficienza, sono state esaminate diverse strategie e tecniche di
implementazione delle strutture dati prima di arrivare alla soluzione
ottimale.  Inoltre il risultato di questa prima fase dello studio ha
portato anche un contributo teorico nell'analisi nel caso medio di
algoritmi dinamici riassunti in due pubblicazioni\cite{SODA96:1,jea:1}.

Nell'ambito del problema del Minimum Spanning Tree sono state messe a
confronto quattro tecniche per l'approccio dinamico al problema:
Sparsification, Frederickson, Il giro di Eulero ed una variante che
impiega gli alberi di Slaetor e Tarjan realizzata ad--hoc per il
problema in questione.

Tutti i risultati raggiunti sono stati considerati di grande interesse
per la comunit\a`a scientifica, tanto che l'intera piattaforma \a`e
stata inserita un package distribuito insieme a LEDA chiamato Leda
Extension Package \cite{ALEX:98} che, sfruttando l'alta
flessibilit\a`a ottenuta mediante tecniche proprie dell'Object
Oriented Programming, punta a rendere concretamente impiegabili i
risultati raggiunti negli ambiti pi\a`u diversi.

Tutte le implementazioni realizzate sono state accuratamente testate 
sia dal punto di vista della prova della correttezza dei risultati che da
quello,  pi\a`u propriamente legato al software engineering, che
riguarda  l'ottimalit\a`a delle strutture dati disegnate. 

Fino ad oggi sono stati sperimentati e valutati almeno 15 algoritmi
noti in letteratura nell'area dei grafi dinamici, mostrando, spesso,
come risultati teorici asintotici esaltanti, possono non trovare
riscontri pratici, mentre altri algoritmi con prestazioni teoricamente
inferiori, ma di pi\`u facile implementazione, possono dar luogo a
migliori risultati nei casi concreti derivanti da problemi reali. I
risultati di tale attivit\`a nell'ambito degli algoritmi su Minimum
Spanning Tree su grafi dinamici, sono stati sintetizzati in \cite{SODA97:2}.

Nel corso degli ultimi anni il sottoscritto, coordinando tesisti e
dottorandi, ha organizzato un gruppo di lavoro grazie al quale \a`e
stato possibile confrontarsi con la qualit\a`a, i ritmi di lavoro ed i
livelli di produttivit\a`a degli altri laboratori impegnati su temi
affini. Tutti i risultati e le esperienze maturate in questo ambito
sono state fotografati in un lavoro di grande sintesi pubblicato in
\cite{acmcs:98}.

\subsection{Animazione di algoritmi e CSCW on the WEB }

Alla luce del crescente interesse che si \a`e sviluppato nell'ambito
del World Wide Web e delle tecnologie introdotte a supporto del
Cooperative Workgroup, per altro molto vicine alle precedenti
esperienze maturate in ambito sistemistico dal sottoscritto,
a partire dal 1994 sono
stati sviluppati numerosi progetti di ricerca che impiegando tali
tecnologie innovative hanno mirato al raggiungimento di un alto grado
di interattivit\a`a su reti geografiche.  Tali temi spaziano dalla
security (Electronic Commerce \cite{ECom:1,ECom:2,CABOTO:98}) alle
adattivit\a`a e tele--teaching \cite{RETIS:97} includendo l'animazione
{\sl over the net} basata sull'impiego di oggetti distribuiti JAVA /
CORBA ed hanno tutti come denominatore comune l'impiego di protocolli
WEB oriented.

In particolare, la maggiore attenzione \a`e stata dedicata alla
ricerca nell'ambito dell'Algorithm Animation visto il ruolo
complementare giocato nei confronti dell'Algorithm Engineering,
rivisitando ed arricchendo tutte le definizioni del problema note, alla luce
delle opportunit\a`a offerte dall'architettura e dai protocolli
pensati per il World Wide Web. Il risultato \a`e racchiuso in un
intero package, denominato CATAI, pronto per l'animazione di qualsiasi
algoritmo che impieghi strutture dati definite a partire dalle
primitive di base di CATAI.  Anche in questo caso la piattaforma
adottata \a`e C++ e LEDA data type. Ci\a`o ha condotto ad una collaborazione
su questi temi con il prof. K. Melhorn che, come ideatore della
piattaforma LEDA, ha intravisto un forte parallelismo tra gli obiettivi
dei rispettivi progetti. Tale sinergia si concretizza nella
possibilit\a`a messa a disposizione da CATAI di supportare l'analisi
ed il debug dei programmi scritti a partire dalle classi di LEDA,
molte delle quali sono gi\a`a state realizzate nelle controparti
animate.  In questo modo aumenta il grado di riutilizzo del lavoro
svolto e, senza ulteriore sforzo, l'utente di LEDA pu\a`o ottenere le versioni
animate delle proprie implementazioni, aumentando la capacit\a`a di
descrivere le strutture dati e la facilit\a`a di messa a punto.  Una
descrizione dettagliata dell'architettura \a`e stata presentata alla
conferenza mondiale IFIP '98. \cite{IFIP:98}.

\subsection{Attivit\a`a legate al dottorato di ricerca}
\begin{description}
\item [$ \bullet $ [ 1987 -- 90]] \hfill \parbox[t]{\boxsize} { A
Parigi durante la permanenza presso il L.I.T.P.  dell' Universit\a`a
Parigi 6 sono state seguite 5 tesi di D.E.S.S. (Diplome d'Etudes
Specialis\a'ees Sup\a'erieures) che hanno portato alla realizzazione
di un rapporto tecnico ed un articolo presentato ad una conferenza
internazionale \cite{ISCIS:1,litp:2} e 2 tesi di terzo ciclo.  Con
alcuni di questi allievi, si sono mantenuti stretti rapporti di
collaborazione scientifica oggetto del paragrafo \ref{parigi} anche
dopo il termine del loro lavoro di tesi e la partenza dal
laboratorio.
}

\item [$ \bullet $ [ 1991 -- 93]] \hfill \parbox[t]{\boxsize} { Nell'
ambito della collaborazione scientifica con il prof. G. Agha ha
coordinato la parte sperimentale della tesi di un suo studente di PhD
dell'Universit\a`a dell'Illinois a Urbana-Champain (U.S.A.),
R. Panwar, che ha svolto uno stage di 60 giorni presso il Dipartimento
di Informatica ed Applicazioni dell'Universit\`a di Salerno
sull'implementazione efficiente di linguaggi Actor-Oriented,
interamente riportato nella sua tesi di PhD e poi oggetto di alcune
pubblicazioni.
}

\item [$ \bullet $ [ 1996 -- 97]] \hfill \parbox[t]{\boxsize}
{nell'ambito del dottorato di Informatica attivato presso il Dipartimento di
Informatica ed Applicazioni ha tenuto un corso di 30 ore, coordinando la
redazione delle tesine finali su problematiche legate
all'InternetWorking con particolari approfondimenti sull'evoluzioni
del protocollo IP da IP 4 a IP 6.
}

\item [$ \bullet $ [ 9/97]] \hfill \parbox[t]{\boxsize}{ Nell'ambito
della scuola nazionale del dottorato di ricerca GII, tenutasi a
Benevento dal 2 al 13 Settembre 1997 il sottoscritto ha tenuto un
corso di tre giorni sull'Analisi sperimentale degli algoritmi in
collaborazione con il prof. G.F. Italiano che ha introdotto il disegno
degli algoritmi. 
}

\item [$ \bullet $ [98 -- 99]] \hfill \parbox[t]{\boxsize}{
Nell'ultimo ciclo del dottorato di ricerca di Salerno, attivato nel
Febbraio del 1998, sta svolgendo il ruolo di tutore scientifico di un
dottorando, con il quali continua ad interessarsi dei temi introdotti
durante la tesi di laurea \cite{IFIP:98}.
}

\end{description}


\subsection{Attivit\a`a di ricerca svolta presso soggetti pubblici e privati}

Sin dai primi anni della sua attivit\a`a scientifica, il sottoscritto
ha sentito l'esigenza di confrontarsi con le principali industrie
dell'area campana o comunque con tutte realt\a`a nazionali ed estere
che avessero affinit\a`a con i temi di interesse scientifici trattati
dal sottoscritto. Questo con la finalit\a`a di avviare proficue
collaborazioni e di individuare quei temi che meglio potessero
coniugare gli interessi scientifici con le reali esigenze del mondo
del lavoro. Tale sforzo \a`e risultato particolarmente utile sia
nell'attivit\a`a didattica, che ha potuto cos\a`\i\ beneficiare di una
lunga serie di feedback da parte dell'industrie software, sia
nell'attivit\a`a scientifica che \a`e rimasta sempre molto legata a
temi concreti e percepibili nel mondo del lavoro . Inoltre da tali
collaborazioni sono derivati grossi vantaggi in termini di risorse che
sono state affidate al Dipartimento per sperimentazione o in termini
di borse di studio per laureandi o post laurea. Questi contatti sono
stati ancora pi\a`u apprezzati con l'introduzione del Diploma di
Laurea in Informatica che espressamente prevede nel suo statuto uno
stretto contatto con il mondo del lavoro da parte dei diplomandi.

Tra le attivit\a`a scientifiche condotte con i partner di maggior
rilevanza in ordine cronologico si citano:

\begin{description}
\item [$ \bullet $ [ 1984 -- 86]] \hfill \parbox[t]{\boxsize} {
ITALTEL S. Maria C.V.  (CE) per una collaborazione sull'introduzione
dei sistemi aperti (UNIX) nella loro catena di produzione e per la
sperimentazione di una linea di apparati trasmissione dati. }

\item [$ \bullet $ [ 1987 -- 90]] \hfill \parbox[t]{\boxsize} { Con il
Laboratoire d'Informatique Th\a`eorique et Programmation
dell'Universit\a`a Parigi 6 ed il Centre National d'Etudes en
T\a'el\a'ecommunications (CNET) \a`e stata tenuta una proficua
collaborazione scientifica i cui risultati sono riportati nel
paragrafo \ref{parigi}
}

\item [$ \bullet $ [ 1985 -- 98]] \hfill \parbox[t]{\boxsize} {
ITALDATA S.p.A. Pianodardine Avellino, gruppo Siemens Data, per
numerosissime collaborazioni su vari temi che andavano dai sistemi
aperti (sperimentazione di stazioni Sinix fornite gratuitamente al
Dipartimento); collaborazione sul progetto finalizzato Calcolo
Parallelo \cite{AICA:1,AICA:2}, numerose tesi di laurea e stage di
formazione, su temi che vanno dal commercio elettronico al CSCW
\cite{ECom:1,ECom:2,CABOTO:98}.
}

\item [$ \bullet $ [ 1990 -- 99]] \hfill \parbox[t]{\boxsize} {
SINTEL Tecnologie gruppo Finmatica per l'introduzione nel processo
produttivo delle tecnologie innovative relative al calcolo distribuito
e pi\a`u in generale alla programmazione ad oggetti {\sl Unified Modeling Language}.
}

\item [$ \bullet $ [ 1996 -- 98]] \hfill \parbox[t]{\boxsize} { Parco
Scientifico e Tecnologico di Salerno, Avellino e Benevento con il
quale sono stati sviluppati molti progetti e sono stati tenuti corsi
divulgativi dell'attivit\a`a scientifica ai soci del PST (Club delle
imprese), anche in questo caso cercando affinit\a`a e spunti di
collaborazione.
}
\end{description}


\section{Attivit\a`a didattica}
\subsection{Carico didattico}
\begin{description}
\item [$ \bullet $ [ 1986 -- 96]] \hfill 
\parbox[t]{\boxsize} { A partire dall'anno accademico 1986--87 e fino
al 1994, ha svolto come carico didattico le esercitazioni del corso
``{\sl Teoria ed applicazioni delle macchine calcolatrici}''
(TAMC). Ha anche organizzato attivamente i laboratori didattici ed ha
partecipato alle commissioni d'esame.

Il contenuto delle esercitazioni si riferiva all'introduzione dei
concetti fondamentali della programmazione e l'insegnamento del
linguaggio C.  Ogni anno sono state coordinate esercitazioni guidate
nei laboratori volte a trovare un'adeguata corrispondenza tra i
concetti astratti insegnati durante le lezioni teoriche e la realt\a`a
delle macchine a disposizione degli studenti.

Nell'anno accademico 1989/90 si \a`e avviata una sperimentazione
didattica volta a ottimizzare la preparazione degli studenti per
ci\a`o che concerne gli approcci alla scrittura ed al debugging di
programmi in ambienti di programmazione evoluti.

Sempre nel contesto del corso di TAMC ha inoltre curato lo sviluppo di
una macchina virtuale equivalente al processore teorico introdotto dal
testo di F. Preparata ``Introduzione alla organizzazione ed alla
progettazione di un elaboratore elettronico''. I programmi nel
linguaggio assembler scritti a lezione potevano cos\a`\i\  essere
eseguiti su  Personal Computer convenzionali.

Nell'anno 1994--95 il contenuto del corso si \`e ulteriormente
evoluto passando dall'architettura astratta del SEC all'architettura
reale del processore MIPS descritta nel testo di Patterson e Hennessy
``Struttura e progetto dei calcolatori. L'interfaccia hardware e
software''.}

\item [$ \bullet $ [ 1996 -- 98]] \hfill 
\parbox[t]{\boxsize} {
Negli anni accademici 1996--98 il carico didattico \a`e stato svolto sul
corso di ``{\sl Sistemi per l'Elaborazione dell'Informazione: Programmazione
su reti}'' tenuto dal prof. G. Persiano. 
In sintonia con il titolo del corso, le esercitazioni sono state
basate prevalentamente su un approccio molto applicativo alla
programmazione di applicazioni che sfruttino il supporto offerto dalla rete,
passando in rassegna prima le primitive offerte dal sistema Unix (socket BSD) poi
i protocolli evoluti basati su RPC (Remote Procedure Call) con cenni a
CORBA e RMI come standard emergenti.}

\item [$ \bullet $ [ 1998 -- 99]] \hfill 
\parbox[t]{\boxsize} { Nell'anno accademico 1998--99 il carico
didattico \a`e stato svolto sul corso di ``{\sl Programmazione I}''
tenuto dai proff. E. Fischetti e M. Napoli.  Anche questo corso \`e
stato improntato alla ricerca di un'adeguata rispondenza tra le
strutture dati astratte insegnate implementazione delle strutture dati
astratti. Il corso tenuto al primo anno, ha consentito una forte
interazione con il corso di ``{\sl Laboratorio I}'' visti i notevoli
punti di contatto.

}
\end{description}

\subsection{Supplenze ex art. 12 Legge 341/90}

\begin{description}
\item \parbox[t]{\posiibox} 
{\bf \raggedright $ \bullet $ [ 1990 -- 92] \par
\medskip
Tecniche Speciali di Elaborazione}
\hfill \parbox[t]{\boxsize} {

Con l'entrata in vigore della legge n. 341 del 19 Novembre 1990, ai
sensi dell'art. 12, ha ricevuto l'affidamento per supplenza del corso
di ``{\sl Tecniche Speciali di Elaborazione}'' presso la Facolt\a`a di
Scienze Matematiche Fisiche e Naturali per gli anni accademici
1990--91 e 1991--92. Il programma del corso, cos\a`\i come concordato
con il Consiglio di Corso di Laurea, trattava molte delle tematiche
relative all'attivit\`a di ricerca svolta.  In particolare venivano
introdotte le principali metodologie per l'approccio al parallelismo
massivo, architetture di elaboratori paralleli a memoria condivisa e a
memoria ripartita, tutte basate su paradigmi computazionali di ``{\sl
message-passing}''. Inoltre ogni anno sono state svolte esercitazioni
guidate in laboratorio basate su programmi sviluppati usando un
linguaggio actor-like ABCL/1 definito da Yonezawa et al.  Questo
linguaggio di programmazione si fonda su concetti molto vicini al
modello originale di G. Agha con un buon approccio metodologico per
quanto concerne la programmazione concorrente orientata agli oggetti.}


\item \parbox[t]{\posiibox} 
{\bf \raggedright $ \bullet $ [ 1992 -- 93] \par
\medskip
Linguaggi Speciali di Programmazione}
\hfill \parbox[t]{\boxsize} {

Per l'anno accademico 1992--93 ha ricevuto l' affidamento per
supplenza del corso di ``{\sl Linguaggi Speciali di
Programmazione}''. Nell' ambito di tale corso, sono stati mantenuti
molti dei contenuti del corso di Tecniche Speciali di Elaborazione con
delle modifiche frutto dell' esperienza degli anni precedenti che
comunque non hanno modificato le caratteristiche essenziali del
corso.}

\item \parbox[t]{\posiibox} 
{\bf \raggedright $ \bullet $ [ 1993 -- 94] \par
\medskip
Sistemi per l'Elaborazione dell'Informazione II}
\hfill \parbox[t]{\boxsize} {

Per l'anno accademico 1993--94 ha ricevuto l'affidamento per supplenza
del corso di ``{\sl Sistemi per l'Elaborazione dell'Informazione
II}''. Nell'ambito di tale corso, fondamentale del secondo anno del
Corso di Laurea in Informatica, ha trattato in modo approfondito tutte
le tematiche inerenti il disegno, lo sviluppo e l'utilizzo dei moderni
sistemi operativi. Il corso comincia con una panoramica sul sistema
operativo Unix come caso di studio per i sistemi time-sharing ed
arriva fino ai sistemi multiprocessore a memoria condivisa (Encore
Multimax - MACH) passando per i cluster di workstation organizzati
intorno ad una rete locale dotati del software necessario per fornire
le primitive di comunicazione e task-management. Per tale corso \a`e
stata redatta una dispensa a supporto degli argomenti non trattati nei
testi adottati \cite{sistemiII}.}

\item \parbox[t]{\posiibox} 
{\bf \raggedright $ \bullet $ [ 1995 -- 98] \par
\medskip
Sistemi per l' Elaborazione dell'Informazione: Reti di Calcolatori I
e II}
\hfill \parbox[t]{\boxsize} {

In seguito alla riorganizzazione del Corso di Laurea in Scienze dell'
informazione, ora rinominato Corso di Laurea in Informatica, ed
all'attivazione di un indirizzo di studio ``{\sl Reti di
Calcolatori}'' ha ricevuto per l'anno accademico 1995--96 l'
affidamento per supplenza dei corsi di ``{\sl Sistemi per l'
Elaborazione dell'Informazione: Reti di Calcolatori}'' parte prima e
parte seconda ciascuno di una unit\`a didattica.  I due programmi si
articolavano intorno al modello di calcolo definito dall'ISO noto come
``{\sl Open System Interconnection}''.  Il primo corso parte da temi
legati alla struttura hardware delle reti locali quali Ethernet e
Token-Ring, per passare via via ai livelli pi\a`u alti per affrontare
problematiche di ottimizzazione (Bridge/Gateway) ed un esempio di
network programming quale NetBIOS.

Il secondo tratta invece aspetti pi\`u specificatamente legati agli
strati software dedicati all'interconnessione di reti ed il
protocollo TCP/IP. Parallelamente al fenomeno dell'{\sl
Internetworking} che si andava via via affermando, sono stati
trattati e messi in evidenza tutti i problemi legati alla
realizzazione di reti geografiche basate su TCP/IP.  Particolare
risalto \`e stato dato agli algoritmi e ai protocolli di routing
progettati per la rete Internet.}
\end{description}

\subsection{Tesi di laurea}
\begin{description}
\item [$ \bullet $ [ 1985 -- 99]] \hfill\parbox[t]{\boxsize} {Circa 80
    tesi di laurea sono state coordinate nell'ambito delle attivit\`a
    didattiche.  Alcune, su tematiche molto vicine agli attuali
    interessi scientifici hanno portato risultati di buon livello.
    Tra queste va senz'altro citata la tesi di laurea di G. Amato,
    coordinata insieme al prof. G.F. Italiano interamente dedicata all'
    implementazione ed all'analisi delle strutture dati descritte da
    Frederickson per il mantenimento di un Minimum Spanning Tree in
    presenza di operazioni di inserimento e rimozione di archi dal
    grafo iniziale.}
\end{description}


\section{Attivit\`a di organizzazione, direzione e coordinamento}

\subsection{Cariche  Istituzionali}
\begin{description}
\item \parbox[t]{\posiibox} 
{\bf \raggedright $ \bullet $ [ 1986 -- 96] \par
\medskip
Responsabile Scientifico Centro di Calcolo Dipartimentale}
\hfill \parbox[t]{\boxsize} {

All' interno delle attivit\a`a dipartimentali, sin dagli anni
successivi alla laurea, ha curato la pianificazione e lo sviluppo
delle strutture di calcolo del Dipartimento di Informatica ed
Applicazioni, interessandosi particolarmente all'importazione ed
all'apprendimento di tutte quelle tecnologie che potevano migliorare
la funzionalit\a`a dei servizi offerti dalla struttura stessa.  In
qualit\`a di responsabile scientifico delle attrezzature informatiche
del Dipartimento, ha coordinato lo sviluppo di tali risorse con
l'obiettivo di disporre di mezzi di calcolo sempre adeguati al
crescente numero degli afferenti ed alle mutevoli esigenze
scientifico/didattiche.  }

\item \parbox[t]{\posiibox}
{\bf  \raggedright $ \bullet $  [ 1990 -- 98] \par
\medskip
Membro della Giunta di Dipartimento}
\hfill \parbox[t]{\boxsize} {

Dal rientro dalla Francia, per poter meglio svolgere il ruolo di
responsabile scientifico delle attrezzature di calcolo, ha fatto parte
della Giunta dipartimentale in qualit\`a di rappresentante dei
ricercatori fino al 1998, collaborando con le direzioni che si sono
succedute negli anni alle scelte tecniche e all'organizzazione dei
laboratori.}

\item \parbox[t]{\posiibox}
{\bf  \raggedright $ \bullet $ [ 1992 -- 99] \par
\medskip
Delegato del Rettore Rete GARR Ateneo Salernitano}
\hfill \parbox[t]{\boxsize} {
Dall'ottobre del 1991, con l'entrata in esercizio della rete GARR
(Gruppo Armonizzazione Reti Ricerca), voluta dal Ministero
dell'Universit\`a e della Ricerca Scientifica e Tecnologica (MURST),
ha coordinato la nascita del polo dell'Ateneo Salernitano, nel ruolo
di responsabile del Polo GARR, delegato del Rettore.  In questo
contesto ha personalmente partecipato alle scelte tecniche del polo
Napoletano ed alla messa a punto delle attrezzature che all'epoca si
basavano su standard ancora non maturi e quindi causa di numerosi
problemi.  Successivamente ha curato la migrazione dell'accesso ad
Internet dal provider I2Unix a GARR, estendendo i servizi telematici a
tutte le componenti dell'Ateneo che ne hanno fatto richiesta.  Da
allora la comunit\a`a degli utenti \a`e diventata via via pi\a`u
numerosa fino a dar vita in modo naturale ad una convenzione stipulata
con l'Ateneo stesso oggetto di un successivo paragrafo.  Tra le varie
attivit\a`a legate a questo ruolo vi \a`e la specifica esigenza di
stabilire le politiche di indirizzamento all'interno del polo (che
risulta un vero e proprio Autonomous System) e di rendere attuali le
politiche stabilite dal GARR. Questo ha comportato la configurazione
ed il mantenimento di tutti i router che si sono succeduti nelle varie
tecnologie, nonch\`e la gestione diretta di tutto il dominio ``{\tt
unisa.it}'' per ci\a`o che riguarda politiche di sicurezza,
instradamento e ottimizzazione del traffico. A questo proposito va
considerata la particolare struttura dell'Ateneo Salernitano che
risulta particolarmente distribuito ed insiste su di un territorio
particolarmente vasto.

In quest'ambito \a`e stato recentemente curata dal sottoscritto la migrazione
alla rete GARR-B, che prevede un notevole aumento della banda a disposizione
ed un'alta qualit\a`a dei servizi, interamente basati su tecnologia ATM.

Successivamente, ha presentato un progetto per un importo previsto di
L. 2.130.000.000 finanziato al 75\% dal MURST, per la realizzazione di
una rete regionale ed in particolare per la creazione delle
infrastrutture all'interno dell'Ateneo per reti multimodali a larga
banda, per fonia, dati ed immagini.  }

\item \parbox[t]{\posiibox}
{\bf  \raggedright $ \bullet $  [ 1993 -- 96] \par
\medskip
Membro Commissione per la risoluzione delle problematiche informatiche
dell' Ateneo Salernitano}
\hfill \parbox[t]{\boxsize} {
Con decreto rettorale \`e stato nominato membro della {\sl Commissione
per la risoluzione delle problematiche informatiche dell'Ateneo
Salernitano}. Tale commissione aveva come incarico quello di
coordinare lo sviluppo delle attrezzature informatiche dell'intero
Ateneo, programmando la nascita di nuovi laboratori e gestendo laddove
possibile la condivisione di risorse tra vari centri con l'intento di
razionalizzare la spesa sia per quanto riguarda l'hardware che per
quanto riguarda la manutenzione e le licenze software.}


\item \parbox[t]{\posiibox}
{\bf \raggedright $ \bullet $  [ 1993 -- 96] \par
\medskip
Membro Commissione aggiudicatrice servizi segreteria studenti}
\hfill \parbox[t]{\boxsize} {
Dopo aver partecipato in seno alla Commissione di Ateneo, all'analisi
delle specifiche ed alla stesura del capitolato speciale d'appalto,
\`e stato nominato membro della commissione aggiudicatrice per la gara
per l'informatizzazione dei servizi di segreteria studenti. La gara
svolta con la modalit\`a di appalto--concorso, ha richiesto la
valutazione di 15 progetti per l'automazione delle segreterie studenti
sia dal punto di vista architetturale che dal punto di vista delle
procedure software.}
\end{description}

\subsection{Organizzazione laboratori scientifici e didattici}

\begin{description}

\item \parbox[t]{\posiibox} 
{\bf \raggedright $ \bullet $ [ 1982 -- 84] \par
\medskip
Laboratorio Personal Computer IBM}
\hfill \parbox[t]{\boxsize} {

In qualit\a`a di tecnico art. 26 ha svolto le sue attivit\a`a avviando
il primo laboratorio di personal Computer, particolarmente dedicato
all'elaborazione delle immagini. Contemporaneamente \a`e stata
realizzata la prima rete locale su tecnologia Token-Ring.  }

\item \parbox[t]{\posiibox} 
{\bf \raggedright $ \bullet $ [ 1984 -- 99] \par
\medskip
Centro di Calcolo Scientico Dipartimentale}
\hfill \parbox[t]{\boxsize} {

Nel 1984, precorrendo le attuali tendenze, si \a`e fatto carico della
migrazione dei sistemi operativi utilizzati dalle macchine del
Dipartimento (VMS per i due VAX 11/750 e 11/785) verso il pi\`u
moderno ed evoluto mondo offerto dai sistemi aperti (Unix BSD).  In
questo contesto \a`e stato realizzato uno dei primi sistemi di posta
elettronica e di news elettroniche attivato in Italia. Le prime
connessioni con l'allora nascente rete Internet avvenivano infatti via
modem su linee telefoniche internazionali, prima con siti francesi e
soltanto nel 1988 con il primo provider italiano I2Unix.
 
Nonostante le difficolt\a`a legate essenzialmente alla assoluta
mancanza di supporto tecnico esterno, il progetto iniziale si \`e
costantemente evoluto fino alla realizzazione dell'attuale rete
dipartimentale, con alta condivisione delle risorse, che conta circa
300 nodi basati su diverse tecnologie che vanno dai personal-computer
ai cluster di workstation quale istanza ideale delle macchine
multiprocessore.  \a`E stato cos\a`i realizzato l'attuale laboratorio
per il calcolo scientifico al quale hanno afferito la maggior parte
delle attivit\a`a scientifiche del Dipartimento, con grande
soddisfazione da parte degli utenti finali.

Contemporaneamente si \a`e avviata una stretta collaborazione con il
personale tecnico amministrativo addetto al laboratorio, che \a`e
stato coordinato al fine di automatizzare tutte le attivit\a`a
essenziali per la corretta erogazione dei servizi.  
}


\item \parbox[t]{\posiibox} 
{\bf \raggedright $ \bullet $ [ 1992 -- 98] \par
\medskip
Laboratorio Didattico Unix}
\hfill \parbox[t]{\boxsize} {

La stessa esperienza nell'ambito dei sistemi aperti ha dato vita al
laboratorio didattico, basato prima su sistemi Ultrix, successivamente
su Open OSF ed infine Digital Unix. Tale sistema \a`e caratterizzato
da un grosso carico di utenti (circa 3000 account consegnati fino ad
oggi), su un numero elevato di postazioni terminali, prima
alfanumerici e successivamente grafici (X-Window).
}


\item \parbox[t]{\posiibox} 
{\bf \raggedright $ \bullet $ [ 1996 -- 99] \par
\medskip
Laboratorio Specialistico Linux/TCFS}
\hfill \parbox[t]{\boxsize} {

Sin dai suoi esordi, particolare interesse \a`e stato dedicato al
sistema operativo Linux, sulle piattaforme Intel e non. In particolare
in collaborazione con il prof. G. Persiano si sono coordinate le
attivit\a`a di un laboratorio specialistico appositamente creato,
all'interno del quale si sviluppano in maniera estremamente
professionale parti del sistema (file system crittografico) che poi
vengono messe a disposizione dell'intera comunit\a`a di utenti sparsa
nel mondo. Finora hanno gravitato attorno al laboratorio ed alle
attivit\a`a ad esso connesso circa 20 studenti, formando cos\a`\i\ un
vero e proprio centro di competenza. Gli utenti del pacchetto
sviluppato sono centinaia e mostrano un interesse crescente per il
servizio offerto partecipando a tutte le frequenti fasi di upgrade e
sviluppo.}


\end{description}

\subsection{Convenzioni ex art. 66 Legge 382/80}

\begin{description}
\item \parbox[t]{\posiibox} 
{\bf \raggedright $ \bullet $ [ 1994 -- 97] \par
\medskip
Sistema Gestionale Opera Universitaria EDISU}
\hfill \parbox[t]{\boxsize} {

Dopo una lunga fase progettuale nei primi mesi del 1995 \`e stata
attivata una convenzione con l'Ente per il Diritto allo Studio
Universitario (EDISU) per la fornitura di tutti i servizi
informatici e per un importo annuo di circa 250.000.000. Tali
servizi in precedenza venivano forniti da una ditta esterna in
services e comprendono tra l'altro la gestione remota di 3 mense
con, rilevazione presenze e gestione del personale, oltre a tutti
gli aspetti per l'automazione della gestione delle graduatorie dei
concorsi per l'assegnazione degli assegni di studio nonch\`e la
gestione finanziaria dell'Ente. Insieme al prof. A. Negro ha
condotto tutta la fase progettuale, disegnando l'architettura
tuttora in funzione.}

\item \parbox[t]{\posiibox} 
{\bf \raggedright $ \bullet $ [ 10/95 -- 99] \par
\medskip
Rete geografica privata e servizi su rete}
\hfill \parbox[t]{\boxsize} {

Nell' Ottobre del 1995 ha stilato i contenuti di una convenzione
triennale stipulata tra il Dipartimento e la societ\`a ``{\sl XCom
Wide Communication S.r.L.}'' per un importo annuo di L.
70.000.000. Di tale convenzione \`e stato responsabile
scientifico.  Tale convenzione ha per obiettivo la sperimentazione
e la prototipizzazione di prodotti ad alta tecnologia che vanno
dalle reti ad alta velocit\`a al software di accesso ad InterNet.
Da tale convenzione e da un accordo con Telecom, \`e stato
realizzata una rete privata che copre tutta la regione Campania ed
alcuni tratti della regione Basilicata per la vendita di accessi
ad InterNet.}

\item \parbox[t]{\posiibox} 
{\bf \raggedright $ \bullet $ [ 4/98 -- 2000] \par
\medskip
Erogazione servizi telematici all'Ateneo Salernitano}
\hfill \parbox[t]{\boxsize} {

Nell'Aprile del 1998 il Direttore del Dipartimento ha firmato una
convenzione con il Rettore dell'Universit\a`a degli studi di Salerno
per la fornitura dei servizi telematici all'intero Ateneo.  Tale
convenzione \a`e il frutto dell'impegno speso nella diffusione di tali
servizi e che hanno visto il Dipartimento in prima linea nel mettere a
disposizione di tutte le componeneti dell'Ateneo le proprie competenze
in questo campo.  In particolare il sottoscritto, ha fattivamente
collaborato, man mano che ne emergeva l'esigenza, con le singole
componenti dell'Ateneo, spesso partendo dal disegno della rete, per
arrivare alla organizzazione di un dominio, ed in particolare alla
definizione dei servizi e delle soluzioni informatiche per le
specifiche realt\a`a.  L'oggetto di tale convenzione \a`e definito in
un Piano Operativo da realizzarsi in due anni per un importo totale di
300.000.000 e verte essenzialmente a realizzare un'organizzazione
capace di erogare in maniera autonoma e centralizzata i servizi che
finora sono stati prestati in maniera volontaristica dal Dipartimento
di Informatica ed Applicazioni. }

\end{description}

\subsection{Pacchetti Software realizzati e distribuiti}

\begin{description}
\item [$ \bullet $ [ 1989]] \hfill \parbox[t]{\boxsize} { Sistema
Prolog completo di ambiente di programmazione realizzato in Common
Lisp. Sono stati ricevuti circa 50 feedback da parte di persone che
hanno scaricato e installato il software.

}
\item [$ \bullet $ [ 1992]] \hfill \parbox[t]{\boxsize} { Sistema per
la gestione dei thread scritto in C ad alte prestazioni e basso tempo
di context switching. Gira sotto Unix BSD ma ormai \a`e stato reso
obsoleto dalle successive modifiche apportate al modello standard e
dall'inclusione nella maggior parte dei Sistemi Operativi. Sono stati
ricevuti circa 150 feedback da parte di persone che hanno scaricato e
installato il software.  }

\item [$ \bullet $ [ 1997]] \hfill \parbox[t]{\boxsize} { Leda
Extension Package, completamente rilasciato solo nel 1998, contiene
l'intera piattaforma realizzata per il testing degli algoritmi su
grafi dinamici e le implementazioni delle strutture dati realizzate.
Sono stati ricevuti circa 200 feedback da parte di persone che hanno
scaricato e installato il software.
} 

\item [$ \bullet $ [ 1997]] \hfill \parbox[t]{\boxsize} { TCFS,
Transparent Cryptographic File System, \a`e il pacchetto pi\a`u
diffuso che conta oltre 500 download e molto interesse da parte degli
utenti.  Interamente scritto in C aggiunge un modulo al sistema
operativo Linux per la gestione di un file system remoto completamente
automatizzata e trasparente per l'utente. Oltre al modulo base \a`e
disponibile anche un pacchetto di utility. Per far fronte alle
frequenti richieste, \a`e stata realizzata una mailing list che viene
utilizzata per far circolare le informazioni sugli ultimi sviluppi,
soluzioni di problemi comuni e suggerimenti di utilizzo in generale, tra gli
utenti interessati.  }

\end{description}

\newpage
\bibliography{personal}
\bibliographystyle{plainrevyr}
%\bibliographystyle{plain}
\addcontentsline{toc}{section}{\refname}



Salerno, l\a`\i\ \thefiledate
\hfill\parbox[t]{5cm}{
  \begin{center}
    firmato\\
    {\sl Giuseppe CATTANEO}
  \end{center}
  \vskip 1cm
  \hrule   
}

\label{lastpage}
\end{document}


